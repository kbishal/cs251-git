\documentclass{report}

\usepackage[ruled]{algorithm2e}
\usepackage[margin = 1in]{geometry}

\title{Movie Reviews}
\author{Aman Deep Singh}
\date{}
\begin{document}
\maketitle

Merge sort\cite{Mergesort} is based on the divide-and-conquer paradigm. Now we give the algorithm to solve the subproblem of sorting $A[p$ .. $r]$. Initially, $p = 1$ and $r = n$, but these values change as we recurse through subproblems.

To sort $A[p$ .. $r]$:
\begin{enumerate}
\item{ Divide Step\\If a given array $A$ has zero element, simply return; it is already sorted. Otherwise, split $A[p$ .. $r]$ into two subarrays $A[p$ .. $q]$ and $A[q + 1$ .. $r]$, each containing about half of the elements of $A[p$ .. $r]$.}
\item{Conquer Step\\
Conquer by recursively sorting the two subarrays $A[p$ .. $q]$ and $A[q + 1$ .. $r]$.}
\item{Combine Step

Combine the elements back in $A[p$ .. $r]$ by merging the two sorted subarrays $A[p$ .. $q]$ and $A[q + 1$ .. $r]$ into a sorted sequence. To accomplish this step, we will define a procedure \textsc{Merge} $(A, p, q, r)$.}
\end{enumerate}

Here I give the psedocode for the Merge Sort Algorithm\\
\begin{algorithm}[H]
\DontPrintSemicolon
\SetKwFunction{FMergeSort}{MergeSort}\SetKwFunction{FMerge}{Merge}
  \SetKwProg{Fn}{Function}{:}{}
  \Fn{\FMergeSort{$A$, $p$, $r$}}{
        \If{$p < r$}{
        	$q \gets (p + r)/2$\;
        	\FMergeSort{A,p,q}\;
        	\FMergeSort{A,q+1,r}\;
        	\FMerge{A,p,q,r}\;
        }
  }
  \;
  \SetKwProg{Fn}{Function}{:}{}
  \Fn{\FMerge{$A$, $p$, $q$, $r$}}{
  		$n1 \gets q - p + 1$\;
  		$n2 \gets r - q$\;
  		let $L[1..n_1 + 1]$ and $R[1..n_2 + 1]$ be new arrays\;
  		\For{$i = 1$ \emph{\KwTo} $n_1$}{
  			$L[i] = A[p + i - 1]$
  		}
  		\For{$j = 1$  \emph{\KwTo}  $n_2$}{
		 	 $R[j] = A[q + j]$
		}
  		$L[n_1 + 1] = \infty$\;
  		$R[n_2 + 1] = \infty$\;
  		$i = 1$\;
  		$j = 1$\;
  		\For{$k = p$ \emph{\KwTo} $r$}{
			\If{$L[i] \leq R[j]$}{
        		$A[k] \gets L[i]$\;
        		$i = i + 1$
       	 }
        	\Else{
        		$A[k] = R[j]$\;
        		$j = j + 1$\;
       	 }
		}
  	}
\caption{MERGESORT\label{MS}}

\end{algorithm}
\pagebreak
\section*{Running Time Analysis}
\begin{itemize}
\item{
\[ T(n) = 2T(n/2) + O(n)\]}
or, \[T(n) = O(nlog(n))\]
\end{itemize}

\medskip
\bibliographystyle{unsrt}
\bibliography{mergesort}
\end{document}